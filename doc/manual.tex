\documentclass[a4paper,12pt]{article}
\usepackage[left=3cm,top=3cm,text={15cm,24cm}]{geometry}
\usepackage[czech]{babel}
\usepackage[utf8]{inputenc}
\usepackage[T1]{fontenc}
\usepackage{lmodern}
\usepackage[scaled=.85]{beramono}

\usepackage{minted}
\usepackage{xcolor}

\renewcommand{\listingscaption}{Kód}
\makeatletter
\renewcommand{\fps@listing}{htp}
\makeatother
\definecolor{code_bg}{gray}{0.95}
\newminted{cpp}{
	bgcolor=code_bg,
	frame=single,
	framerule=0pt
}
\newmintinline{cpp}{
	bgcolor=code_bg
}
\begin{document}

\begin{titlepage}
	\begin{center}
		{
			\LARGE\textsc{Vysoké učení technické v~Brně}\\
			\Large\textsc{Fakulta informačních technologií}\\
		}
		\vspace{\stretch{0.3}}
		{
			\textbf{\Huge DNS Resolver}\\
			\vspace{3mm}
			\LARGE ISA Projekt\\
		}
		\vspace{\stretch{0.618}}
	\end{center}
	{\large Listopad 2019 \hfill Jan Chaloupka (xchalo16)}
\end{titlepage}
\tableofcontents
\newpage

\section{Úvod do problematiky}
Hello world \cppinline{if(size > 5) return "bagr";}
\section{Návrh aplikace}

\section{Popis implementace}
Jep
\begin{listing}
\caption{Vytvoření DNS dotazu typu MX na adresu fit.vutbr.cz}
\begin{cppcode}
Dns::Message request;
request.RecursionDesired = opt.RecursionDesired;

Dns::Question qst;
qst.Class = Dns::CLASS_IN;
qst.Type = Dns::TYPE_MX;
qst.Name = "fit.vutbr.cz";
request.Question.push_back(qst);

Dns::Bytes bytes = request.ToBytes();
\end{cppcode}
\end{listing}
\section{Informace o programu}

\section{Návod na použití}

\end{document}
