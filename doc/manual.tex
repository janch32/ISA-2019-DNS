\documentclass[a4paper,12pt]{article}
\usepackage[left=3cm,top=3cm,text={15cm,24cm}]{geometry}
\usepackage[czech]{babel}
\usepackage[utf8]{inputenc}
\usepackage[T1]{fontenc}
\usepackage{lmodern}
\usepackage{hyperref}
\usepackage[scaled=.85]{beramono}

\usepackage{minted}
\usepackage{xcolor}

\renewcommand{\listingscaption}{Kód}
\makeatletter
\renewcommand{\fps@listing}{htp}
\makeatother
\definecolor{code_bg}{gray}{0.95}
\newminted{cpp}{
	bgcolor=code_bg,
	frame=single,
	framerule=0pt
}
\newmintinline{cpp}{
	bgcolor=code_bg
}

\begin{document}

\begin{titlepage}
	\begin{center}
		{
			\LARGE\textsc{Vysoké učení technické v~Brně}\\
			\Large\textsc{Fakulta informačních technologií}\\
		}
		\vspace{\stretch{0.3}}
		{
			\textbf{\Huge DNS Resolver}\\
			\vspace{3mm}
			\LARGE ISA Projekt\\
		}
		\vspace{\stretch{0.618}}
	\end{center}
	{\large Listopad 2019 \hfill Jan Chaloupka (xchalo16)}
\end{titlepage}
\tableofcontents
\newpage

\section{Úvod do problematiky}
\subsection{Struktura DNS paketu}
Detailní informace o struktuře DNS paketů jsou popsány v dokumentu RFC 1035\footnote{Celé znění dokumentu RFC 1035: \url{www.ietf.org/rfc/rfc1035}}, ze kterého se v tomto projektu vychází. Navíc se v projektu využívají záznamy a struktury související s IPv6, které jsou definovány v dokumentu RFC 3596\footnote{Celé znění dokumentu RFC 3596: \url{www.ietf.org/rfc/rfc3596}}

Pro komunikaci využívá DNS protokol UDP, v základu na portu 53. Není tak zaručeno, že zpráva dorazí bez chyby. Samotný DNS paket je rozdělen na pět základních částí. Detailně je DNS zpráva popsána v RFC 1035, sekce 4.

\subsubsection{Hlavička}
Vždy přítomná část zprávy. Obsahuje základní informace jako ID, stavový kód, možnosti dotazu a údaje o počtu záznamů v jednotlivých sekcí.

\subsubsection{Dotazy}

\subsubsection{Odpovědi}

\subsubsection{Autority}

\subsubsection{Další záznamy}


\subsection{Reverzní DNS dotaz}
Reverzní záznam slouží k přiřazení doménového jména k IP adrese. Tento záznam je označený jako \texttt{PTR} a obsahuje v datové sekci doménové jméno.
Pro získání tohoto záznamu je třeba poslat dotaz typu \texttt{PTR} s IP adresou ve speciálním tvaru jako doménové jméno.

\subsubsection{Převedení IPv4 adresy do PTR formátu}
Adresa typu IPv4 se převede obrácením pořadí jednotlivých oktetů a přidáním řetězce \texttt{.in-addr.arpa.} na konec (RFC1035 sekce 3.5).
\paragraph{Příklad:}\texttt{203.99.78.77.in-addr.arpa.} odpovídá IP adrese \texttt{77.78.99.203}

\subsubsection{Převedení IPv6 adresy do PTR formátu}
Převedení IPv6 adresy funguje na podobném principu jako v4 adresy. Adresa se rozdělí po půl bajtech (jeden hexa znak) a zapíše se v plné délce v obráceném pořadí. Znaky jsou dděleny tečkou. Nakonec se za adresu přidá řetězec \texttt{.ip6.arpa.} (RFC3596 sekce 2.5).
\paragraph{Příklad:}Adrese \texttt{2001:67c:1220:809::93e5:917} odpovídá zápis \texttt{7.1.9.0.5.e.3\linebreak .9.0.0.0.0.0.0.0.0.9.0.8.0.0.2.2.1.c.7.6.0.1.0.0.2.ip6.arpa.}

\section{Návrh aplikace}
Výsledná aplikace je psaná v jazyce C++, konkrétně ve standardu C++11. 



\section{Popis implementace}

\subsection{Seznam použitých knihoven}

Aplikace využívá následující standardní knihovny jazyka C++\footnote{Seznam standardních C++ knihoven: \url{en.cppreference.com/w/cpp/header}}: 
\texttt{algorithm},
\texttt{cerrno},
\texttt{cstdint},
\texttt{cstdlib},
\texttt{cstring},
\texttt{ctime},
\texttt{iomanip},
\texttt{iostream},
\texttt{list},
\texttt{sstream},
\texttt{stdexcept},
\texttt{string} a
\texttt{vector}.

Kromě toho je v projektu využito několika POSIX knihoven jazkya C\footnote{Seznam C POSIX knihoven: \url{pubs.opengroup.org/onlinepubs/9699919799/idx/head.html}}, zejména pro práci s BSD sockety. Použité C POSIX knihovny jsou:
\texttt{arpa/inet.h},
\texttt{netdb.h},
\texttt{netinet/in.h},
\texttt{sys/socket.h},
\texttt{sys/time.h} a
\texttt{unistd.h}.


\section{Informace o programu}

\section{Návod na použití}

\end{document}
